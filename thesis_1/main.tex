%%%%%%%%%%%%%%%%%%%%%%%%%%%%%%%%%%%%%%%%%
% Masters/Doctoral Thesis 
% LaTeX Template
% Version 2.5 (27/8/17)
%
% This template was downloaded from:
% http://www.LaTeXTemplates.com
%
% Version 2.x major modifications by:
% Vel (vel@latextemplates.com)
%
% This template is based on a template by:
% Steve Gunn (http://users.ecs.soton.ac.uk/srg/softwaretools/document/templates/)
% Sunil Patel (http://www.sunilpatel.co.uk/thesis-template/)
%
% Template license:
% CC BY-NC-SA 3.0 (http://creativecommons.org/licenses/by-nc-sa/3.0/)
%
%%%%%%%%%%%%%%%%%%%%%%%%%%%%%%%%%%%%%%%%%

%----------------------------------------------------------------------------------------
%	PACKAGES AND OTHER DOCUMENT CONFIGURATIONS
%----------------------------------------------------------------------------------------

\documentclass[
11pt, % The default document font size, options: 10pt, 11pt, 12pt
%oneside, % Two side (alternating margins) for binding by default, uncomment to switch to one side
english, % ngerman for German
singlespacing, % Single line spacing, alternatives: onehalfspacing or doublespacing
%draft, % Uncomment to enable draft mode (no pictures, no links, overfull hboxes indicated)
%nolistspacing, % If the document is onehalfspacing or doublespacing, uncomment this to set spacing in lists to single
%liststotoc, % Uncomment to add the list of figures/tables/etc to the table of contents
%toctotoc, % Uncomment to add the main table of contents to the table of contents
%parskip, % Uncomment to add space between paragraphs
%nohyperref, % Uncomment to not load the hyperref package
headsepline, % Uncomment to get a line under the header
%chapterinoneline, % Uncomment to place the chapter title next to the number on one line
%consistentlayout, % Uncomment to change the layout of the declaration, abstract and acknowledgements pages to match the default layout
]{MastersDoctoralThesis} % The class file specifying the document structure

\usepackage[utf8]{inputenc} % Required for inputting international characters
\usepackage[T1]{fontenc} % Output font encoding for international characters

\usepackage{mathpazo} % Use the Palatino font by default

\usepackage[backend=bibtex,style=authoryear,natbib=true]{biblatex} % Use the bibtex backend with the authoryear citation style (which resembles APA)
\usepackage{tablefootnote}

\addbibresource{example} % The filename of the bibliography

\usepackage[autostyle=true]{csquotes} % Required to generate language-dependent quotes in the bibliography
\usepackage{graphicx}

\usepackage{hyperref}

\usepackage{fourier} 
\usepackage{array}
\usepackage{makecell}

\renewcommand\theadalign{bc}
\renewcommand\theadfont{\bfseries}
\renewcommand\theadgape{\Gape[4pt]}
\renewcommand\cellgape{\Gape[4pt]}


%----------------------------------------------------------------------------------------
%	MARGIN SETTINGS
%----------------------------------------------------------------------------------------

\geometry{
	paper=a4paper, % Change to letterpaper for US letter
	inner=2.5cm, % Inner margin
	outer=3.8cm, % Outer margin
	bindingoffset=.5cm, % Binding offset
	top=1.5cm, % Top margin
	bottom=1.5cm, % Bottom margin
	%showframe, % Uncomment to show how the type block is set on the page
}

%----------------------------------------------------------------------------------------
%	THESIS INFORMATION
%----------------------------------------------------------------------------------------

\thesistitle{Deep Learning for Anomaly Detection} % Your thesis title, this is used in the title and abstract, print it elsewhere with \ttitle
\supervisor{Prof. Dr. Thomas \textsc{Hanne}} % Your supervisor's name, this is used in the title page, print it elsewhere with \supname
\examiner{} % Your examiner's name, this is not currently used anywhere in the template, print it elsewhere with \examname
\degree{Master of Science in Business Information Systems} % Your degree name, this is used in the title page and abstract, print it elsewhere with \degreename
\author{Kevin \textsc{Saner}} % Your name, this is used in the title page and abstract, print it elsewhere with \authorname
\addresses{} % Your address, this is not currently used anywhere in the template, print it elsewhere with \addressname

\subject{Biological Sciences} % Your subject area, this is not currently used anywhere in the template, print it elsewhere with \subjectname
\keywords{} % Keywords for your thesis, this is not currently used anywhere in the template, print it elsewhere with \keywordnames
\university{Fachhochschule Nordwestschweiz} % Your university's name and URL, this is used in the title page and abstract, print it elsewhere with \univname
\department{\href{https://www.fhnw.ch/en/about-fhnw/schools/business}{School of Business}} % Your department's name and URL, this is used in the title page and abstract, print it elsewhere with \deptname
\faculty{\href{}{}} % Your faculty's name and URL, this is used in the title page and abstract, print it elsewhere with \facname

\AtBeginDocument{
\hypersetup{pdftitle=\ttitle} % Set the PDF's title to your title
\hypersetup{pdfauthor=\authorname} % Set the PDF's author to your name
\hypersetup{pdfkeywords=\keywordnames} % Set the PDF's keywords to your keywords
}

\begin{document}
	


\frontmatter % Use roman page numbering style (i, ii, iii, iv...) for the pre-content pages

\pagestyle{plain} % Default to the plain heading style until the thesis style is called for the body content

%----------------------------------------------------------------------------------------
%	TITLE PAGE
%----------------------------------------------------------------------------------------

\begin{titlepage}
\begin{center}

\vspace*{.06\textheight}
{\scshape\LARGE \univname\par}\vspace{1.5cm} % University name
\textsc{\Large Master Thesis Proposal}\\[0.5cm] % Thesis type

\HRule \\[0.4cm] % Horizontal line
{\huge \bfseries \ttitle\par}\vspace{0.4cm} % Thesis title
\HRule \\[1.5cm] % Horizontal line
 
\begin{minipage}[t]{0.4\textwidth}
\begin{flushleft} \large
\emph{Author:}\\
\href{}{\authorname} % Author name - remove the \href bracket to remove the link
\end{flushleft}
\end{minipage}
\begin{minipage}[t]{0.4\textwidth}
\begin{flushright} \large
\emph{Supervisor:} \\
\href{}{\supname} % Supervisor name - remove the \href bracket to remove the link  
\end{flushright}
\end{minipage}\\[3cm]
 
\vfill

\large \textit{A thesis proposal submitted in fulfillment of the requirements\\ for the degree of \degreename}\\[0.3cm] % University requirement text
\textit{in the}\\[0.4cm]
\deptname\\[2cm] % Research group name and department name
 
\vfill

{\large \today}\\[4cm] % Date
%\includegraphics{Logo} % University/department logo - uncomment to place it
 
\vfill
\end{center}
\end{titlepage}

%----------------------------------------------------------------------------------------
%	DECLARATION PAGE
%----------------------------------------------------------------------------------------

\begin{declaration}
\addchaptertocentry{\authorshipname} % Add the declaration to the table of contents
\noindent \section*{Declaration of Authenticity}

I the undersigned declare that all material presented in this paper is my own work or fully and specifically acknowledged wherever adapted from other sources. I understand that if at any time it is shown that I have significantly misrepresented material presented here, any degree or credits awarded to me on the basis of that material may be revoked. I declare that all statements and information contained herein are true, correct and accurate to the best of my knowledge and belief. This paper or part of it have not been published to date. It has thus not been made available to other interested parties or examination boards.\\

Olten, 21st of December 2021\\
 
\noindent Name:\\
\rule[0.5em]{25em}{0.5pt}

\noindent Signed:\\
\rule[0.5em]{25em}{0.5pt} % This prints a line for the signature

\end{declaration}

\cleardoublepage

%----------------------------------------------------------------------------------------
%	QUOTATION PAGE
%----------------------------------------------------------------------------------------

\vspace*{0.2\textheight}

\noindent\enquote{\itshape Your brain does not manufacture thoughts. Your thoughts shape neural networks.}\bigbreak

\hfill Deepak Chopra

%----------------------------------------------------------------------------------------
%	ABSTRACT PAGE
%----------------------------------------------------------------------------------------

\begin{abstract}
\addchaptertocentry{\abstractname} % Add the abstract to the table of contents
In the age of Big Data, data analysis becomes ever more important. To analyse the data, many researchers nowadays focus on artificial intelligence. Artificial intelligence does not rely on labour-intensive feature engineering like the traditional machine learning or statistical models. Therefore, the use of AI, such as neural networks, can save a lot of development time. Two widely used architectures of neural networks are the Convolutional Neural Networks and the Recurrent Neural Networks. A Convolutional Neural Network is generally used when a task is related to image recognition, whereas Recurrent Neural Networks are used for the prediction of time series. Recently an approach was proposed to analyse time series data with Convolutional Neural Networks. The strengths and weaknesses of this approach, however, are currently unknown and are further investigated in this paper. To examine the usefulness, the practically relevant use case of anomaly detection was chosen. Within the scope of this work, different approaches on anomaly detection, that employ convolutional or recurrent neural networks are investigated. Since the architectures should be compared regarding their performance, ways to evaluate the performances are assessed. After elaborating the methodology applied in this work, it is described how the hyper-parameters were determined to make the models comparable. As a main part of this work, three experiments on different datasets are conducted. The datasets used, vary in complexity and contain different types of anomalies. The first experiment was carried out on a synthetic dataset with synthetic anomalies. In the second experiment a real-world dataset with synthetic anomalies was used. Finally, an official benchmark dataset was employed in the third experiment. On the obtained results the two architectures are assessed according to their usefulness for anomaly detection. Further, it is classified how helpful deep learning is for forecasting time series and detecting anomalies. At last, the insights of this work are presented together with suggestions for future research.     

\end{abstract}

%----------------------------------------------------------------------------------------
%	ACKNOWLEDGEMENTS
%----------------------------------------------------------------------------------------

\begin{acknowledgements}
\addchaptertocentry{\acknowledgementname}
 I would first like to thank my thesis advisor Prof. Dr. Thomas Hanne, lecturer in business information systems at Fachhochschule Nordwestschweiz for the useful comments, remarks and engagement through the learning process of this master thesis. 
 
 
 I would also like to thank the experts at my workplace, especially Manuel Dominguez and Benjamin Volland, who, in various discussions, provided new ideas and ways to approach the questions presented throughout this thesis.  
\end{acknowledgements}

%----------------------------------------------------------------------------------------
%	LIST OF CONTENTS/FIGURES/TABLES PAGES
%----------------------------------------------------------------------------------------

\tableofcontents % Prints the main table of contents

\listoffigures % Prints the list of figures

\listoftables % Prints the list of tables

%----------------------------------------------------------------------------------------
%	ABBREVIATIONS
%----------------------------------------------------------------------------------------

\begin{abbreviations}{ll} % Include a list of abbreviations (a table of two columns)

\textbf{DNN} & \textbf{D}eep \textbf{N}eural \textbf{N}etwork\\
\textbf{CNN} & \textbf{C}onvolutional \textbf{N}eural \textbf{N}etwork\\
\textbf{RNN} & \textbf{R}ecurrent \textbf{N}eural \textbf{N}etwork\\
\textbf{GRU} & \textbf{G}ated \textbf{R}ecurrent \textbf{U}nit Network\\
\textbf{LSTM} & \textbf{L}ong \textbf{S}hort \textbf{T}erm \textbf{M}emory\\
\textbf{ROC} & \textbf{R}eceiver \textbf{O}perating \textbf{C}haracteristics\\
\textbf{AUC} & \textbf{A}rea \textbf{U}nder the \textbf{C}urve\\
\textbf{ReLu} & \textbf{R}ectified \textbf{L}inear \textbf{U}nit\\
\textbf{MAE} & \textbf{M}ean \textbf{A}bsolute \textbf{E}rror \\
\textbf{FCN} & \textbf{F}ully \textbf{C}onvolutional \textbf{N}etworks\\


\end{abbreviations}

%----------------------------------------------------------------------------------------
%	PHYSICAL CONSTANTS/OTHER DEFINITIONS
%----------------------------------------------------------------------------------------

%\begin{constants}{lr@{${}={}$}l} % The list of physical constants is a three column table

% The \SI{}{} command is provided by the siunitx package, see its documentation for instructions on how to use it

%Speed of Light & $c_{0}$ & \SI{2.99792458e8}{\meter\per\second} (exact)\\
%Constant Name & $Symbol$ & $Constant Value$ with units\\

%\end{constants}

%----------------------------------------------------------------------------------------
%	SYMBOLS
%----------------------------------------------------------------------------------------

%\begin{symbols}{lll} % Include a list of Symbols (a three column table)

%$a$ & distance & \si{\meter} \\
%$P$ & power & \si{\watt} (\si{\joule\per\second}) \\
%Symbol & Name & Unit \\

%\addlinespace % Gap to separate the Roman symbols from the Greek

%$\omega$ & angular frequency & \si{\radian} \\

%\end{symbols}

%----------------------------------------------------------------------------------------
%	DEDICATION
%----------------------------------------------------------------------------------------

%\dedicatory{For/Dedicated to/To my\ldots} 

%----------------------------------------------------------------------------------------
%	THESIS CONTENT - CHAPTERS
%----------------------------------------------------------------------------------------

\mainmatter % Begin numeric (1,2,3...) page numbering

\pagestyle{thesis} % Return the page headers back to the "thesis" style

% Include the chapters of the thesis as separate files from the Chapters folder
% Uncomment the lines as you write the chapters

% Chapter 1

\chapter{Introduction} % Main chapter title

\label{1.} % For referencing the chapter elsewhere, use \ref{Chapter1} 

%----------------------------------------------------------------------------------------

% Define some commands to keep the formatting separated from the content 
\newcommand{\keyword}[1]{\textbf{#1}}
\newcommand{\tabhead}[1]{\textbf{#1}}
\newcommand{\code}[1]{\texttt{#1}}
\newcommand{\file}[1]{\texttt{\bfseries#1}}
\newcommand{\option}[1]{\texttt{\itshape#1}}


%----------------------------------------------------------------------------------------


With the rise of the Internet of Things (IoT) and ever more sensors, gadgets and smart devices like smartwatches for fall detection or blood pressure monitoring, or fridges for temperature protective control in use, the amount of available data steadily increases \parencite{Alansari2018}. Simultaneously, the possibilities to use the data to draw conclusions increases. This data is generally used to draw conclusions such as failure of a system or a medical issue, such as a heart attack. These events typically occur very rarely \parencite{Hauskrecht2007}. However, when the number of instances of each class is approximately equal, most machine learning algorithms function best. Problems occur when the number of instances of one class greatly exceeds the number of instances of the other. This issue is very popular in practice, and it can be observed in a variety of fields such as fraud detection, medical diagnosis, oil spillage detection, facial recognition, and so on \parencite{Thabtah2020}. The task of identifying the rare item, event or observation is often referred to as anomaly detection. Typically, the anomalous item translates to problems such as bank fraud or medical problems. Often, the anomaly does not adhere to the common statistical definition of an outlier. Therefore, many outlier detection methods (in particular unsupervised methods) fail on such data \parencite{Hodge2004}

A special disciplin in anomaly detection is to find the anomaly in a time series. The anomaly detection problem for time series is usually formulated as finding outlier data points relative to some standard or usual signal. Time series anomaly detection plays a critical role in automated monitoring systems. It is an increasingly important topic today, because of its wider application in the context of the Internet of Things (IoT), especially in industrial environments. The most popular techniques to find the anomalies are:

\begin{itemize}
	\item Statistical Methods
	\item Support Vector Machines
	\item Clustering 
	\item Density-based Techniques
	\item Neural Networks 
\end{itemize}

Currently Neural Networks are regarded the cutting-edge research. Although first approaches to use artificial neural networks exist since 1969. They are popular in research for only about 15 to 20 years, which is why scientist generally expect further improvements on this kind of technology. This promising outlook is why this research paper also purely focusses on Neural Networks for anomaly detection. Especially Recurrent Neural Networks and Convolutional Neural Networks are investigated and compared. 

\section{Definitions}
Following, the most important terms in the context of anomaly detection using neural networks are elaborated and defined. 

\subsection{Univariate, Bivariate and Multivariate Data}
Time series data investigation poses a special disciplin. Generally anomalies in time series are harder to detect for traditional statistical models, since the possibility of long term dependencies exist. Time series data comes in different forms. It is distinguished between univariate, bivariate and multivariate data. Univariate involves the analysis of a single variable while bivariate and multivariate analysis examines two or more variables. One significant difference between time-series and other datasets is that the observations are dependent not only on the components d, but also on the time feature n. Thus, time-series analysis and the statistical methods employed are largely distinct from methods employed for random variables, which assume independence and constant variance of the random variables. Time-series are important to data analysts in a variety of fields such as the economy, healthcare and medical research, trading, engineering, and geophysics. These data are used for forecasting and detecting anomalies.

\subsubsection{Univariate Data}
There is only one variable in this type of data. Because the information only deals with one variable that changes, univariate data analysis is the simplest type of analysis. It is not concerned with causes or relationships, and the primary goal of the analysis is to describe the data and identify patterns.

\subsubsection{Bivariate Data}
This type of data involves two different variables. This type of data analysis is concerned with causes and relationships, and the goal is to determine the relationship between the two variables.

\subsubsection{Multivariate Data}
Multivariate data is defined as data that contains three or more variables. It's similar to bivariate, but there are more dependent variables meaning there is not only one variable that influences the observed beahavior (independent variable) but several. The methods for analyzing this data are determined by the objectives to be met. Regression analysis, path analysis, factor analysis, and multivariate analysis of variance are some of the techniques. Data collected from several sensors installed in a car is an example of a multivariate time-series.

\subsection{Neural Networks}

An Artificial Neural Network (ANN) with several layers between the input and output layers, is known as a Deep Neural Network (DNN). Neural networks come in a variety of shapes and sizes, but they all have the same basic components: neurons, weights and functions. These components work in a similar way as human brains and can be trained just like any other machine learning algorithm.

\subsubsection{Neuron}
Artificial neurons represent the smallest building blocks of neural networks. A neuron usually receives separately weighted inputs which it sums. The sum is then passed through the activation function to calculate the output of the neuron. When training a neural network, the input weights are adjusted by the optimizer function to improve accuracy of the given task e.g. classification.

\subsubsection{Layer}
In neural networks three different kind of layers are distinguished. There are input, output and hidden layers. A layer can be described as a collection of neurons. All layers between the input and output layer are called hidden layers. In the input layer data is fed into the neural network. The output of the hidden layer is calculated by taking the weighted sums of input and passing it through the activation function. Typically, a more complex problem requires more hidden layers to accurately calculate the output. In the output layer the final result e.g. a classification is produced. Figure \ref{fig:layers} how a simple Neural Network with just one hidden layer could look like.


\begin{figure}[h]
	\centering
	\includegraphics[scale=0.3]{Figures/layers}
	\decoRule
	\caption[Layers]{Input, Hidden and Output Layers \parencite{DennyBritz2015}}
	\label{fig:layers}
	%https://blog.csdn.net/mydear_11000/article/details/51087980
\end{figure}



\subsubsection{Optimizer Function}


\section{Background}

Following, it is explained how different kinds of neural networks work and what they are used for.

\subsection{Neural Networks for Anomaly Detection} \label{NN}

Out of the three most popular neural network architectures, convolutional neural networks (CNN), recurrent neural networks (RNN) and deep neural networks (DNN), only RNN are typically used for anomaly detection in time series. RNNs have built-in memory and are therefore able to anticipate the next value in a time series based on current and past data. Classic or vanilla RNNs can theoretically keep track of arbitrary long-term dependencies in input sequences. There, however, is a computational issue: when using back-propagation to train a vanilla RNN, the back-propagated gradients can "vanish" or "explode" due to the computations involved in the process, which use finite-precision numbers. Because LSTM units allow gradients to flow unchanged, RNNs using LSTM unit or Gated Recurrent units (GRU) partially solve the vanishing gradient problem and therefore drastically improve accuracy \parencite{Hochreiter1997}.

Specially to mention in this context are LSTM (Long-Short Term Memory) and GRU (Gated Recurrent Units). Both achieved outstanding performance when used for tasks such as unsegmented, connected handwriting recognition, speech recognition and anomaly detection in network traffic or IDSs (intrusion detection systems) \parencite{JunyoungChung2014}

\subsubsection{LSTM} \label{LSTM}
LSTM was first proposed in 1997 by Schmidhuber and Hochreiter \parencite{Hochreiter1997}. The initial version to the LSTM unit consisted of a cells, input and output gates. In 1999, the LSTM architecture was improved by introducing a forget gate and therefore allowing the LSTM to reset its own state \parencite{Gers2000}. LSTM is used in a supervised training approach, that means it tries to predict a predefined state taking the past and the current state. If the predicted state differs from the expected state, the weights of the different gates are adjusted using an optimizer algorithm such as gradient descent. Figure \ref{fig:LSTM} shows how the gates and the cell are arranged. The cell represents the memory of the LSTM. In simple words, the LSTM works as follows to predict a new value: 

\begin{enumerate}
	\item Forget Gate: Obsolete information is removed from the cell state.
	\item Input Gate: New information is added to the cell state
	\item Output Gate: The new information and the cell state are added to make the new prediction.
	\item The new cell state is propagated to the next LSTM unit  
\end{enumerate}
     
\begin{figure}[h]
	\centering
	\includegraphics[scale=0.5]{Figures/LSTM}
	\decoRule
	\caption[LSTM]{Gates and Cell of LSTM  \parencite{MichaelPhi2018}}
	\label{fig:LSTM}
\end{figure}

% from https://towardsdatascience.com/illustrated-guide-to-lstms-and-gru-s-a-step-by-step-explanation-44e9eb85bf21

%\subsubsection{GRU}
% maybe not necessary

\subsubsection{CNN}
In contrast to RNNs Convolutional Neural Networks are generally used for image classification. CNNs work as feature extractors and are able to recognize patterns . CNNs use layers that are not fully connected, to reduce complexity (compare to \ref{NN}). In a CNN, a set number of neurons forms a filter. These filters or kernels are the actual feature extractors. A filter may represent a line or pattern (see Figure \ref{fig:filter}) \parencite{LeCun1998}.

\begin{figure}[h]
	\centering
	\includegraphics[scale=0.7]{Figures/filter}
	\decoRule
	\caption[Kernel]{Example of a Filter used in CNN}
	\label{fig:filter}
\end{figure}

To detect whether, a feature is occurrent in a picture, the filter is gradually moved over the picture in so called strides. In every step (stride) the dot-product between the filter and the part of the picture is calculated. The results of the operations are stored in activation maps.  The greater the dot-product the more alike are the filter and the section of the image. Training the network hereby refers to determining the shapes of these filters.
Other typical features of a CNN are the pooling layers. The pooling layers reduce the amount of computation necessary. The most commonly used pooling technique is max-pooling and works as shown in Figure \ref{fig:pooling}.

\begin{figure}[h]
	\centering
	\includegraphics[scale=0.2]{Figures/pooling}
	\decoRule
	\caption[Pooling]{Example of max-pooling}
	\label{fig:pooling}
\end{figure}

The idea of max-pooling is to only keep the maximum value of an activation map. In the orange region 7 represents the maximum value, so it is kept while the other values are discarded \parencite{RichStureborg2019}.

In 2019, Wen and Keys proposed to use CNN also for anomaly detection in time series since it shares many common aspects with image segmentation. A univariate time series is therefore viewed as a one-dimensional image.\\ 


% https://towardsdatascience.com/conv-nets-for-dummies-a-bottom-up-approach-c1b754fb14d6


\subsection{Transfer Learning}
%https://builtin.com/data-science/transfer-learning
%Section about transfer learning\\
The reuse of a previously trained model on a new problem is known as transfer learning. It is currently very popular in deep learning because it can train deep neural networks with a relatively small amount of data. This is particularly useful in the field of data science, as most real-world problems do not provide millions of labeled data points to train complex models.
In transfer learning, the knowledge of an already trained machine learning model is applied to a different but related problem. For example, if a classifier was trained to predict if an image contains a backpack, the model's experience could be applied to recognize other objects such as sunglasses \parencite{NiklasDonges2020}.

\section{Problem Statement}

Defining a ground truth is one of the most difficult aspects of time series anomaly detection. Determining when anomalous behavior begins and ends in time series is a difficult task, as even human experts are likely to disagree in their assessments. Furthermore, there is the question of what constitutes a useful detection when detecting anomalies in time series.
In the past, RNN have successfully been used for anomaly detection (e.g. [\parencite{Malhotra2015}; \parencite{Fan2016}]. Therefore, designs for various use cases are well researched. RNN are well suited for the task, however, take a long time to train due to the complexity of how a single unit is designed (see \ref{LSTM}). In comparison CNN are not as complex and therefore, generally take less time to train. However, CNNs are generally used for image recognition and were only very recently used for anomaly detection in time series. It is therefore mostly unknown what designs are applicable for successful anomaly detection in time series data. %is this true?
While RNN are able to deal with multivariate data by design, a classical CNN requires design changes to be able to deal with multivariate data. Wen and Keys \parencite*{Wen2019} proposed to use a special kind of U-net, an improved version of a Fully Convolutional Neural Network \parencite{Ronneberger2015}.
Further, a CNN is not capable to analyze streaming data so it relies on segmentation of the data. These data segments are called snapshots. In order to not miss any data points, the frequency of taking these snapshots should be at least as high as the length of snapshot so that every time point is evaluated by the model at least once. However, for better performance it might prove beneficial to use a higher frequency which means every point is evaluated various times by the model \parencite{Wen2019}. The proposed design change and the fact that every point is evaluated multiple times, increases complexity and evaluation time and therefore counteract the architectural advantage of CNN compared to a RNN. 
When designing a neural network many parameters have to be chosen, this applies to both mentioned types of Neural Networks. For example, when designing a CNN, the number of layers, the activation function(s) of a single neuron and the optimizer function have to be chosen. Additionally, when using CNN for time series data the length and frequency of the snapshots have to be determined. Similarly, when designing a RNN also the number of layers and the optimizer function have to be determined. Because the basic building blocks of both networks types are very different it is difficult to fairly compare the complexity of two architecture approaches. Another important parameter which applies to both network types is the number of epochs for which the networks are trained. Through the epochs the training time is determined. In order to compare the two types of neural networks, two networks of similar complexity have to be designed. With equal training time the performance of both can be compared and evaluated. A RNN is therefore only set up as benchmark while the main goal of this research project is to clarify whether CNNs are really useful and propose an advantage over RNNs when applied on time series data for anomaly detection.

%transfer learning is missing

\section{Thesis Statement}

Convolutional Neural Networks are superior to Recurrent Neural Networks when looking for anomalies in time series data regarding training time and complexity.

\subsection{Subquestions}

\begin{itemize}
	\item How does a CNN for univariate and multivariate data need to be designed for successful anomaly detection in time series data?
	\item What advantages and disadvantages arise when using a CNN compared to a RNN for anomaly detection in univariate and multivariate time series?
	\item What parameter settings are crucial for a fair performance comparison between RNN and CNN? 
	\item Optional: How does transfer learning affect the performance of CNN compared to RNN in anomaly detection in time series?
\end{itemize}

 
\subsection{Research Objectives}

Following the research objectives of this paper are defined.

%always start with a verb ... to test, to determine

\begin{enumerate}
	\item Determine what design changes a CNN requires to detect anomalies in time series data.
	\item Determine how the CNN should be designed for the comparison with a RNN
	\item State the advantages and disadvantages of the chosen CNN architecture.
	\item Define parameters which allow a fair comparison of CNN and RNN
\end{enumerate}

\subsection{Limitations}

Recently there have been approaches that combine CNN and RNN into a hybrid network for tasks such as handwriting recognition or video-based emotion recognition \parencite{Dutta2018} \parencite{Fan2016}. However, this paper only compares pure CNN and RNN, and does investigate a hybrid approach.

% AUC and ROC not explained
% Test and training not explained 

\subsection{Significance}

Until now, time series data was almost only approached with RNNs. This paper should answer the question whether CNNs propose a valid alternative and even propose some advantages over RNNs. The paper will answer the fundamental question whether research should channel efforts to further investigate CNNs for anomaly detection in time series data or whether no benefits can be discovered and research is better to focus on other areas. 

\subsection{Chapter Overview}


%----------------------------------------------------------------------------------------

% Chapter Template

\chapter{Literature Review} % Main chapter title

\label{2.} % Change X to a consecutive number; for referencing this chapter elsewhere, use \ref{ChapterX}

%----------------------------------------------------------------------------------------
%	SECTION 1
%----------------------------------------------------------------------------------------


\section{Literature Review}

\subsection{RNN for Anomaly Detection}

 -- state of the art

\subsection{CNNs}

\subsection{CNN for time series data}

\subsection{Parameter Settings for fair comparison}

\subsection{Transfer Learning}

\section{Research Gap}
 
% Chapter 3

\chapter{Research Methodology} % Main chapter title

\section{Introduction}
This section describes which research approach was chosen and why. Further, it elaborates, how the research approach is implemented and what work is done in the corresponding sections. 

\section{Research Design}
As research methodology experimental research was chosen. Experimental research typically focuses on systematically testing a hypothesis. It is often applied to research fields such as physics and chemistry but also psychology. In this research project, the hypothesis to test is formulated as the thesis statement (see section \ref{thesisstatement}.\\
\\
Experimental research knows five process steps. These are:

\begin{itemize}
	\item Awareness of the Problem
	\item Design of Experiments
	\item Experiments
	\item Evaluation
	\item Conclusion
\end{itemize}

In the following, it is outlined what will be done in the different process steps and how it is going to help answering the research question and test the thesis statement. Figure \ref{Thesis Map} illustrates the different steps.

\begin{figure}[h]
	\centering
	\includegraphics[scale=0.5]{Figures/Thesis Map}
	\decoRule
	\caption[Thesis Map]{Thesis Map \parencite{own}}
	\label{Thesis Map}
\end{figure}

\newpage
\subsection{Awareness of the Problem - Literature Review}
In this work, the literature review consists of two parts, background information presented in Section \ref{background} and the part Related Literature presented as Chapter \ref{relatedLiterature}. The reason for this split is to give insight on how neural networks and especially the derived architectures such as CNN and RNN work. A basic understanding of these two architectures is required to comprehend the problem and thesis statement presented in Sections \ref{Problem} and \ref{thesisstatement}. 

Further, the part Related Literature should deliver insights on how CNN and RNN are applied to detect anomalies. Where helpful, the publications investigated not only focus on detecting anomalies but also on related fields such as classification or prediction of time series. The investigated knowledge fields should serve as suggestions on how anomaly detection models can be set up. Next, the possibilities of transfer learning are examined. Since artificial intelligence experiences a boom in recent years, and neural networks are used more frequently, the possibilities offered by transfer learning will increase. Even more important, transfer learning is able to tackle one of the fundamental problems in anomaly detection. It enables to learn a performant model, even on a small data set. 

At last, to be able to answer the research question it is investigated how the different architectures, RNN and CNN, can be compared and what measures are important. 

\subsection{Experimental Design}
In this section, it is outlined how the experiments are designed. It is described how and why datasets are selected as well as which architecture principles are followed when designing the neural networks.

\subsubsection{Data Selection}
\textcolor{red}{
As described in Section \ref{anomalies}, there are different kinds of anomalies, which are more or less difficult to detect. In the Section Design of Experiments suitable data sets are proposed. At least, three different data sets are selected to conduct experiments.}
 
Further, the selected data sets have a direct impact on how the experiments have to be designed. Here the decision has to be made, whether an unsupervised or supervised approach is applied. A supervised approach hereby requires a data set where the anomalies are labelled. In contrast, an unsupervised approach can investigate any time series data for anomalies, but it is difficult to validate the achieved results.

\subsubsection{Setup of Experiments}
In the Section Setup of Experiments the general setup of the experiments is elaborated. Chapter \ref{relatedLiterature} showed that there are various approaches on how to detect anomalies. Anomalies can either be directly classified by the neural network or detected via an anomaly threshold. This decision has to made based on the chosen data set, since only in a supervised approach a network can be designed to directly classify whether an anomaly is present. In Section \ref{SetupOfExperiments} a suitable method to answer the research questions is proposed. In addition, the advantages of the proposed setup, especially with an outlook on how the achieved results can be compared, are elucidated.   
Further, in Section \ref{SetupOfExperiments}, all global parameters are defined. Global parameters are parameters that need to be defined manually rather than learned and are valid for all experiments. An example of such a global parameter could be the optimizer function and its corresponding parameters.

\subsection{Experiments}
Finally, in the Section Experiments the proposed experimental setup is implemented. The experiments are only conducted on multivariate time series, since Braei and Wagner \parencite*{Braei2020} already issued a comprehensive study comparing different approaches for anomaly detection on univariate time series. Since transfer learning was not part of the referred study, the experiments on transfer learning can, however, also be done on univariate time series. 

In the Chapter \ref{Experiments}, the design of the neural networks is described. It includes the determination of the basic architecture, which is specific to the proposed data set, further, it also includes a lot of tuning work to figure out the best parameters for each neural network. To test the performance of the neural networks, also a baseline classifier is established. The baseline classifier is of a very simple nature. Inspired by the trivial null classifier \footnote{The trivial null classifier always predicts the majority class and thus represents the minimum precision every useful model should surpass.} the established baseline will figure as the benchmark to beat for the deep learning approaches. 

\subsection{Evaluation - Results Analysis}
The Section Results Analysis is dedicated to the examination of the previously achieved results. The results are compared using predefined metrics such as training time, inference time and F1-Score. Looking at these metrics helps to compare the neural network architectures and to determine whether CNN are actually superior to RNN when applied for anomaly detection. Comparing the neural networks to the baseline algorithm further gives some insight on how useful neural networks are in general for anomaly detection.

\subsection{Conclusion}
When comparing the neural networks on an anomaly detection task, it is expected that no architecture is overall superior. For example, the high accuracy of RNN comes with the drawback of long training and inference times whereas a CNN with possibly lower accuracy outperforms the RNN especially on inference time. Thus, the decision which architecture to use depends on the use case. Therefore, as conclusion a set of recommendations, that shows what architecture is best suited for a certain use case, is compiled.



\chapter{Experimental Design}
In this Chapter, the datasets are selected, general design ideas are explained and established, global parameters are defined and at last the experiments and their goals are described.

\section{Tools}
The following Sections provide insight on what tools, such as software and hardware, is used to conduct the experiments.

\subsection{Hardware}
All experiments are conducted on a HP Probook equipped with a Intel Core i7 with 2.8GHz and 32GB of RAM installed. No additional graphics card was used.

\subsection{Software}
All necessary code is written in R (Version 4.1.0). To design the neural networks, the library Keras (Version 2.4.0) together with the Tensorflow (Version 2.5.0) backend is used.


\section{Datasets}
Following, it is motivated how and why the proposed data sets are used.

\subsection{Problems of Existing Benchmarks} \label{Problems of Existing Benchmarks}
In order to find out, which neural network architecture is better suited for anomaly detection, first, suitable datasets have to be evaluated. Most of the papers on anomaly detection test on one of the popular benchmark datasets such as the ones created by Numenta, Yahoo, NASA, or Pei's Lab. These benchmark datasets are, however, declared as flawed by Wu and Keogh \parencite{YEAR}. Wu and Keogh state that the benchmark datasets suffer from at least one of the following flaws:

\begin{enumerate}
	\item \textbf{Triviality:} Surprisingly, a sizable proportion of the problems in the benchmark datasets are trivial to solve. Triviality is hereby defined as follows: An anomaly can be found with just one line of code.
	\item \textbf{Unrealistic Density:} This flaw refers to too many anomalies in the dataset or at least in a certain region, whereas in a real world dataset the anomalous data points make up a portion of just above 0 percent.   
	\item \textbf{Mislabeled Ground Truth:} The data in all of the benchmark datasets appears to be mislabeled, with both false positives and false negatives. This is significant for a number of reasons. The majority of anomaly detectors work by computing statistics for each subsequence of some length. They may, however, place their computed label at the beginning, end, or middle of the subsequence. If caution is not exercised, an algorithm may be penalized for reporting a positive just to the left or right of a labeled region.
	\item \textbf{Run-to-failure Bias:} Because many real-world systems are run-to-failure, there is often no data to the right of the last anomaly. Therefore, a naïve algorithm that labels the last point as an anomaly has a very good chance of being correct.
\end{enumerate}

In their work, Wu and Keogh, introduced the UCR Time Series Anomaly Datasets as new benchmark, that avoids the problems listed above. However, at the start of this research project the datasets were not publicly available. Because the search for a dataset, that does not suffer from the above mentioned flaws, would be too time-consuming, the decision was taken to partly engineer own datasets.
\newline
\subsection{Anomalies}
The neural networks should be used to detect various types of anomalies, in order to test their ability to recognize them. Foorthuis \parencite{YEAR} compiled, in an extensive literature review, a study on the different types of anomalies. The anomalies were divided into different categories, of which foremost the quantitative multivariate aggregate anomalies are relevant for this research project, especially a) to f) (see figure \ref{fig:Anomaly_types}). These types of anomalies typically occur in time series data, that is composed by sensor data. Examples of such data could be temperature measurements or Electrocardiograms.

\begin{figure}[h]
	\centering
	\includegraphics[scale=0.6]{Figures/series_anomaly}
	\decoRule
	\caption[Quantitative Anomalies]{Quantitative Anomalies \parencite{Foorthuis}}
	\label{fig:Anomaly_types}
	%https://arxiv.org/ftp/arxiv/papers/2007/2007.15634.pdf
\end{figure}

\subsection{Dataset Selection}
In the following subsections, it is proposed how and why the datasets are selected for the different experiments.

\subsubsection{1. Dataset}
The dataset, which should be used for the first experiment will be of synthetic nature. It consists of various cyclic patterns. In a second step, the dataset is enriched with anomalies. This way, two dataset are produced. The dataset without anomalies is used for an unsupervised learning approach whereas the dataset with labelled anomalies is used for a supervised approach. Further information, on how the dataset is created, what it looks like and what the anomalies look like can be found in Section \ref{dataset1} 

\subsubsection{2. Dataset}
The second dataset, which is used for anomaly detection, should consist of real data. To make sure, that the requirements, mentioned in section \ref{Problems of Existing Benchmarks} are met, the anomalies are embedded manually into the dataset. 

\subsubsection{3. Dataset}
As third dataset, one of the existing benchmark datasets should be used. Despite their obvious flaws, it is still considered useful to validate the previously achieved results on an official benchmark. Further, this gives insights into the overall usefulness of the proposed neural network architectures.

\subsection{Split of Datasets}
With all datasets the classical train, validation and test dataset approach is chosen. For the supervised approach, the training data is enriched with anomalies, whereas for the unsupervised approach a "clean" dataset is used. The final evaluation is done on a test dataset, which is the same for all approaches.


\section{Setup of Experiments} \label{SetupOfExperiments}
The following section explains how the different experiments are conducted in detail. When desinging the experiments, the focus is put on comparability rather than optimally tuned neural networks.  Further, it is shown how the datasets and anomalies were engineered.

The following subsections give information on the chosen setups of the experiments that apply to all experiments. Further, global parameters are defined. Global parameters are defined manually and are the same for all experiments. Parameters, which do not fall in this category such as number of neurons in a layer, are specified in Chapter \ref{Experiments} in the section belonging to the respective experiment.

\subsection{Supervised Learning}
The supervised learning approach refers to training the neural network on a labelled dataset. The dataset used for training already has the anomalies embedded. The task of finding the anomalies can also be described as a classification task, where the neural networks classifies a sequence as normal or anomalous. In such a case binary crossentropy and a sigmoid activation function are used as loss function and last layer activation fuction. The aforementioned combination means that in the last layer a logistic regression is done, where a threshold is determined to classify the sequences into normal and anomalous.

Supervised classification tasks are used and function well when sufficient samples of all classes are available. In anomaly detection, this is generally not the case, as the anomalous is per definition underrepresented. In the literature (e.g. \parencite{Wen2019}), however, the supervised approach is still successfully used. There are two explanation for this: First, as explained in Section \ref{Problems of Existing Benchmarks}, the density of anomalies is unrealistic and second, the anomalies are of the same kind and always look very similar, so a neural network is able to learn the pattern of the anomaly. In the experiments it is investigated, how the neural networks react when presented with anomalies that are similar to the ones in the training data, but also to previously unseen anomalies of the same kind (e.g. level shift). It is expected, that any neural network will fail when there are not enough similar anomalies in the training data.

\subsection{Unsupervised Learning}
The unsupervised learning approach refers to the training of a neural network on a dataset that is free of anomalies. The neural network merely learns the cyclic pattern of the data. When learning the pattern the loss function applied is Mean Absolute Error (MAE), so the learning task is a regression, which itself is supervised. The actual anomaly detection hereby is done in a second step. As proposed in section \ref{CNN on univariate series}, the anomaly detector module, which is also used in this experiment, calculates the Euclidean distance between predicted and actual value, where a large value corresponds to an anomaly.

A unsupervised approach is the more reliable choice when trying to detect anomalies, because the model does not need to know what anomalies might look like. However, setting up this approach is more time consuming, especially when dealing with multivariate time series, as for every variable, a separate detector module and corresponding threshold needs to be set up. It is yet to be expected, that both neural network architectures, CNN and RNN, will outperform their supervised counterparts. 

\subsection{Neural Networks}
The following Sections describe principles that will be used in the design of the neural networks. The described principles and their corresponding paramters are used in all experiments. Parameters that are specific to the experiment are specified in the corresponding section in the Chapter \ref{Experiments}. 

\subsubsection{Normalization}
Before the data is fed into the neural network it is normalized. Normalization ensures that the magnitude of the values that a feature assumes are more or less the same. Therefore, the mean of each time series is subtracted of each time series, and divided by the standard deviation. As normalization parameters, the mean and standard deviation was used for all, test, validation and training dataset.  
%https://towardsdatascience.com/why-data-should-be-normalized-before-training-a-neural-network-c626b7f66c7d

\subsubsection{Activation Function}
When designing the neural networks, as activation function generally "ReLu" (Rectified Linear Unit) is used. The function is non-linear and basically just returns the input if it is bigger than 0 and otherwise 0. This function is widely used, because of its simplicity and generally yields good results with little computation expenses.  

\subsubsection{Optimizer}
As optimizer, the often used default choice in machine learning, ADAM (Adaptive Moment Estimation), with the proposed default values, is applied \parencite{AUTHOR,YEAR}. ADAM updates the learning rate when training, making it faster than other optimizers such a Gradient Descent. On the downside, however, ADAM uses a lot memory for a given batch size and is found to generalize poorly in late stages of training.
%https://www.deeplearning.ai/ai-notes/optimization/ 

\subsection{Experiments}
In the following, it is suggested how the experiments on the 3 different datasets are conducted.

\subsubsection{1. Experiment}
In a first experiment the learning abilities of RNN and CNN are compared. It is investigated how useful the architectures are in a supervised and in an unsupervised setup. This experiment gives general insight on which setups and approaches work under which conditions. As the anomalies embedded are similar in the training and test set, the supervised approaches should yield good results, whereas the unsupervised approaches, given the simplicity of the dataset, are not expected to miss any anomalies.

\subsubsection{2. Experiment}
The second experiment is conducted on a more challenging dataset. Also the embedded anomalies are of a more challenging nature. All approaches that have been found successful in Experiment 1, are investigated further on the new dataset. Since the dataset consists of more, partially dependent, variables, and more challanging cyclic patterns the architectures used in Experiment 1 have to be extended for example by adding additional layers. 

\subsubsection{3. Experiment}
In the third experiment, a dataset that has already been used in the field of anomaly detection should be used. The neural network architecture types CNN and RNN are applied to the chosen dataset. First, this shows which approach is better suited and second, the achieved results can be compared to already existent results to verify the overall performance of the chosen anomaly detection methods.

\subsection{Results}
As results three metrics are reported. First and most important, is the F1-Score which gives insight on the models ability to recognize anomalies. Second and third, the training and inference time are reported.



  
\chapter{Experiments}
 
\section{Experiment 1}
The first experiment was conducted on a fully synthetic dataset. Supervised and unsupervised learning approaches were used to detect the anomalies. 

\subsection{Dataset}
The dataset, that was created for this first task consisted of five variables. The dataset was created under the assumption, that one measurement was drawn per hour on totally 2000 days, resulting in 48000 datapoints. The variables all follow a cyclic pattern shown in figure \ref{fig:synthetic data} but are not dependent on each other. 

\begin{figure}[h]
	\centering
	\includegraphics[scale=0.7]{Figures/synthetic data}
	\decoRule
	\caption[Synthetic Dataset]{Synthetic Dataset \parencite{own}}
	\label{fig:synthetic data}
\end{figure}

In a second step the dataset was enriched with six different kinds of anomalies. The anomalies embedded into the dataset are of type deviant cycle, temporary change and level shift. Figure \ref{fig:anomalies} shows examples of the embedded anomalies. The same kind of anomalies were embedded into the training and the test dataset.

\begin{figure}[h]
	\centering
	\includegraphics[scale=0.7]{Figures/Anomalies}
	\decoRule
	\caption[Synthetic Anomalies]{Synthetic Anomalies \parencite{own}}
	\label{fig:anomalies}
\end{figure}

For the supervised learning approach the dataset was also labelled. In each case the whole day was marked as an anomaly. In total 30 anomalous days were embedded into the dataset, this corresponds to 1.5 percent of anomalous datapoints.

\subsection{Neural Networks}
In the first experiment, a CNN was tested against a RNN in a supervised and also in an unsupervised fashion. As RNN, the type GRU was chosen. As some first attempts showed, that it showed sufficient results with improved computation time compared to the more complex LSTM. 

For the supervised learning approach, the chosen architecture can be seen in table \ref{Tab:Supervised Learning1}. 

\begin{table}[h]
\caption{Supervised Learning}
	\begin{center}
		\begin{tabular}{ | c | c | c | c |}
			\hline
			\thead{} & \thead{Input} & \thead{NN-Architecture} & \thead{Output} \\
			\hline
			\thead{CNN} &  120 past datapoints  & \makecell{2 1D-Convolutional Layers \\ 2 Max-Pooling Layers \\ 1 Dense Layer}  & \makecell{1 Dense Layer \\ with Sigmoid Activation}   \\
			\hline
			\thead{RNN} &  120 past datapoints  & \makecell{2 GRU Layers \\ 1 Dense Layer}  & \makecell{1 Dense Layer \\ with Sigmoid Activation}  \\
			\hline
		\end{tabular}
		\label{Tab:Supervised Learning1}
	\end{center}
\end{table}

 
The architecture used for the unsupervised learning approach, displayed in table \ref{Tab:Unupervised Learning1}, looked very similar. The main difference between the two architectures was that the CNN was not built as a sequential model. Instead of having one output layer, the CNN has five parallel output layers, each predicting a time series \footnote{More detailed summaries of the models can be found in appendix \ref{AppendixA}}.     

\begin{table}[h]
	\caption{Unupervised Learning}
	\begin{center}
		\begin{tabular}{ | c | c | c | c |}
			\hline
			\thead{} & \thead{Input} & \thead{NN-Architecture} & \thead{Output} \\
			\hline
			\thead{CNN} &  120 past datapoints  & \makecell{2 1D-Convolutional Layers \\ 2 Max-Pooling Layers }  & \makecell{ 5 Dense Layers with \\ 1 regression outputs}   \\
			\hline
			\thead{RNN} &  120 past datapoints  & \makecell{2 GRU Layers}  & \makecell{ 1 Dense Layers with \\ 5 regression outputs}  \\
			\hline
		\end{tabular}
	\label{Tab:Unupervised Learning1}
	\end{center}
\end{table}

\subsection{Results}

Inference time and F1-Score are, for all models, calculated on the same test dataset. As the initial dataset, it consists of 48000 datapoints, with 30 anomalous days embedded. For the supervised approach using a RNN, no F1-Score is reported, as the model, just like the trivial null classifier, always predicted no anomaly. Since the anomalies span over a whole day, for all approaches, the results were averaged per day. Resulting in normal or anomalous days. The reported F1-Score was finally calculated on this data. 


\begin{table}[h]
	\caption{Results}
	\begin{center}
		\begin{tabular}{ | c | c | c | c |}
			\hline
			\thead{} & \thead{F1-Score} & \thead{Training Time} & \thead{Inference Time} \\
			\hline
			\thead{CNN Supervised} &  0.991388  & 169s  & 422s   \\
			\hline
			\thead{RNN Supervised} &  -  & 967s   & 952s   \\
			\hline
			\thead{CNN Unsupervised} & 0.9989817  & 129s   & 435s   \\
			\hline
			\thead{RNN Unsupervised} &  0.9979613  & 300s   & 793s   \\
			\hline
			\thead{Trivial Null Classifier} &  0.9924204  & -  & -  \\
			\hline
		\end{tabular}
		\label{Tab:Results1}
	\end{center}
\end{table}

From the table \ref{Tab:Results1}, it can be seen that the supervised learning approach takes longer to train. This can be explained by the fact that a CNN must first learn the patterns and only then begins to recognise the anomalies. Further, it shows that despite promising results when training, the trivial null classifier achieves a higher F1-Score than the supervised CNN approach. 
The best score was achieved with a CNN applied in an unsupervised fashion. The CNN reported just one false negative and a false positive, compared to one false negative and two false positives of the RNN.

\newpage

\section{Experiment 2}
In a second experiment, a real dataset was used as base. The anomalies were embedded manually. Since, the unsupervised approach with RNN in Experiment 1 proved useless it was not included in the second experiment. Since the anomalies in training and test set were the same in Experiment 1, the CNN supervised was able to recognize some of the anomalies. In the second approach, it should be checked if this is approach could be of any use, if the anomalies consist of a hitherto unknown pattern.

\subsection{Dataset}
The dataset used was derived from the "Appliances Energy Prediction Dataset" available on the UCI Machine Learning Repository. The dataset consists of 9 room temperatures and corresponding humidity levels, energy in use of ligth and appliances, two random variables as well as six variables containing weather information. The dataset consists of 19735 datapoints, where 6 datapoints are drawn per hour. Of the available variables only 10 variables were used for the anomaly detection task. The variables used are 5 room temperatures, energy use, outside temperature, air pressure and wind speed. The variables were selected because they show a dependency. For example, a high outside temperature and energy use result in high temperatures in the different rooms. Figure \ref{fig:temp_dataset} provides an insight on the used dataset. 

\begin{figure}[h]
	\centering
	\includegraphics[scale=0.6]{Figures/temp_dataset}
	\decoRule
	\caption[Temperature Dataset]{Appliances Energy Prediction Dataset \parencite{Own or UCI???}}
	\label{fig:temp_dataset}
\end{figure}

Again, the anomalies were embedded manually into the dataset. The anomalies are of type level shift, deviant cycles, variaton change and distribution-based aggregate anomaly. The anomalies were embedded into the room temperature variables and the energy use variable. Figure \ref{fig:temp_anomalies} shows examples of the mentioned anomalies.

\begin{figure}[h]
	\centering
	\includegraphics[scale=0.7]{Figures/temp_anomalies}
	\decoRule
	\caption[Temperature Dataset Anomalies]{Examples of Embedded Anomalies \parencite{Own}}
	\label{fig:temp_anomalies}
\end{figure}

\subsection{Neural Networks}

\subsection{Results}


\section{Experiment 3}
 
\chapter{Results Analysis}



\chapter{Conclusion}

%----------------------------------------------------------------------------------------
%	THESIS CONTENT - APPENDICES
%----------------------------------------------------------------------------------------

\appendix % Cue to tell LaTeX that the following "chapters" are Appendices

% Include the appendices of the thesis as separate files from the Appendices folder
% Uncomment the lines as you write the Appendices

% Appendix A

\chapter{Model Summaries} % Main appendix title

\label{AppendixA} % For referencing this appendix elsewhere, use \ref{AppendixA}

\section{Summary of Models of Experiment 1}

\subsection{Supervised Learning}
\begin{figure}[h]
	\centering
	\includegraphics[scale=0.5]{Figures/summary_CNN_class_syn}
	\decoRule
	\caption[Synthetic Anomalies]{Summary of CNN \parencite{own}}
	\label{fig:summary_CNN_class_syn}
\end{figure}

\begin{figure}[h]
	\centering
	\includegraphics[scale=0.5]{Figures/summary_GRU_class_syn}
	\decoRule
	\caption[Synthetic Anomalies]{Summary of GRU \parencite{own}}
	\label{fig:summary_GRU_class_syn}
\end{figure}

\newpage
\subsection{Unsupervised Learning}
\begin{figure}[h]
	\centering
	\includegraphics[scale=0.5]{Figures/summary_CNN_pred_syn}
	\decoRule
	\caption[Synthetic Anomalies]{Summary of CNN \parencite{own}}
	\label{fig:summary_CNN_pred_syn}
\end{figure}

\begin{figure}[h]
	\centering
	\includegraphics[scale=0.5]{Figures/summary_GRU_pred_syn}
	\decoRule
	\caption[Synthetic Anomalies]{Summary of GRU \parencite{own}}
	\label{fig:summary_GRU_pred_syn}
\end{figure}

\section{Summary of Models of Experiment 2}

\subsection{Supervised Learning}
%\begin{figure}[h]
%	\centering
%	\includegraphics[scale=0.5]{Figures/TBF}
%	\decoRule
%	\caption[Synthetic Anomalies]{Summary of CNN \parencite{own}}
%	\label{fig:CNN_classifier_house_temp}
%\end{figure}

\subsection{Unsupervised Learning}
\begin{figure}[h]
	\centering
	\includegraphics[scale=0.5]{Figures/summary_CNN_pred_house_temp}
	\decoRule
	\caption[Synthetic Anomalies]{Summary of CNN \parencite{own}}
	\label{fig:summary_CNN_pred_syn}
\end{figure}

\begin{figure}[h]
	\centering
	\includegraphics[scale=0.5]{Figures/summary_LSTM_pred_house_temp}
	\decoRule
	\caption[Synthetic Anomalies]{Summary of GRU \parencite{own}}
	\label{fig:summary_GRU_pred_syn}
\end{figure}
\include{Appendices/AppendixB}
%\include{Appendices/AppendixC}

%----------------------------------------------------------------------------------------
%	BIBLIOGRAPHY
%----------------------------------------------------------------------------------------

\printbibliography[heading=bibintoc]

%----------------------------------------------------------------------------------------

\end{document}  
